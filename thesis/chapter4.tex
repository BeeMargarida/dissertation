\chapter{Problem Statement} \label{chap:problem_statement}

\section*{}

This chapter describes the problem, as it can be seen in Section \ref{sec:current_issues}. In Section \ref{sec:disiderata} it is presented the wanted features for the proposed solution and in Section \ref{sec:scope} the scope of the project is defined. Section \ref{sec:research_questions}contains the research questions to be answered by this dissertation. The experimental methodology is outlined in Section \ref{sec:exp_meth}. Chapter \ref{sec:planning} contains a Gantt chart with a planning of this dissertation. Finally, this chapter is summarized by Section \ref{sec:stat_summary} with an overview of the topics mentioned before.

\section{Current Issues}\label{sec:current_issues}

Chapter \ref{chap:sota} contains several solutions that provide decentralized architecture in visual programming tools applied to the internet of things paradigm. However, some of this tools are specific to a certain paradigm, like Smart Cities or industry. \textcolor{red}{Check this after SOTA}
We can define the problem in these issues:
\begin{enumerate}
    \item \textbf{Discovery of computation capabilities}: the current work lacks the automatic discovery of the computational capabilities of the devices in the network. This information is normally manually introduced by the developer.
    \item \textbf{Leveraging devices in the network}:
    \item \textbf{Inferring computational capabilities}: 
    \item \textbf{Detecting non-availability}:
\end{enumerate}

\section{Desiderata}\label{sec:disiderata}

\textcolor{blue}{explain what a desiderata is; explain each desiderata point}

\noindent
\textbf{D1: Infer computational capabilities of devices connected}

\noindent
\textbf{D2: Decomposition and partition of the computation}

\noindent
\textbf{D3: Convert computational tasks into runnable code}

\noindent
\textbf{D4: Provide self-adaptation of the system}

\section{Scope}\label{sec:scope}

%\section{Use Cases}\label{sec:use_cases}

\section{Research Questions}\label{sec:stat_research_questions}

\section{Experimental Methodology}\label{sec:exp_meth}

\section{Planning}\label{sec:planning}

\textcolor{blue}{Gantt chart}

\section{Summary}\label{sec:stat_summary}
