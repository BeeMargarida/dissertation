\chapter{Problem Statement} \label{chap:problem_statement}

\section*{}

\minitoc \mtcskip \noindent
This chapter describes the problem tackled by this dissertation, as well as the features it proposes to develop and the hypothesis it aims to validate. \sectionref{sec:current_issues} addresses several limitations present in the current state of the art that remain to be addressed. \sectionref{sec:desiderata} details a set of propositions for the prototype and methodology that will be implemented. \sectionref{sec:scope} defines the scope of the project and \sectionref{sec:main_hypothesis} contains the hypothesis and attributes that will be used as validation. The experimental methodology is outlined in \sectionref{sec:exp_meth}. Finally, this chapter is summarized in \sectionref{sec:stat_summary} with an overview of the topics mentioned before.

\section{Current Issues}\label{sec:current_issues}

\chapterref{chap:sota} contains several solutions that attempt to provide a decentralized architecture in visual programming tools applied to the Internet-of-Things paradigm. However, these tools solve specific problems or make assumptions regarding the scale of the system and the constraints of the devices.
\sectionref{sec:decentralized_sota_conclusion} contained several limitations in the current solutions which remain to be addressed, namely:
\begin{enumerate}
    \item \textbf{Leveraging idle computation power available}: Fog Computing introduces a decentralized solution, one that can be applied to Node-RED by distributing the computational tasks across the edge devices. A decentralized systems not only takes advantage of the constrained devices present in Fog and Edge tiers, but also allow for more resiliency of the system to failures by removing the centralized instance's single point of failure. Although there is still a centralized element that orchestrates the decentralization, it is not essential once an assignment is made and all devices are functioning.
    \item \textbf{Communication of computational capabilities}: the decomposition and assignment of tasks to devices requires information about the capabilities of the device to make an informed and optimized choice. Therefore, it is necessary for each device to communicate its capabilities to the orchestrator.
    \item \textbf{Code generation of sub-flows}: most devices found in Edge tiers are not capable of running any Linux-based or more complex systems. Therefore, it is necessary to take advantage of their capabilities with a basic method of computation, a script. Most constrained devices are capable of executing firmware communication via HTTP and MQTT as well as the execution of scripts. Thus the generation of scripts from \textit{nodes} is the best way to fully utilize each device's resources. 
    \item \textbf{Provide self-adaption of the system}: devices can fail or recover, as well as the connection between them or even the network. It is important for the system to detect these changes and adapt to them at run-time, orchestrating itself to always keep functioning. 
\end{enumerate}

\section{Desiderata}\label{sec:desiderata}
Our works proposes a methodology, as well as a prototype that addresses the above limitations. Such tool would fullfil the following desiderata:

\begin{description}
    \item \textbf{D1: Communicate computational capabilities of connected devices}, so that this information can be sent to an orchestrator that, based in this data, will decompose the total computation workload.
    \item \textbf{D2: Automatic decomposition and partition of computation}, so that the total computational requested can be distributed through all the devices in the network, using information about the computational capabilities and availability of the devices in the network.
    \item \textbf{D3: Convert computational tasks into runnable code}, so that they can be executed in edge and fog devices, taking into consideration their specific constraints.
    \item \textbf{D4: Provide self-adaptation of the system}, so that it can adapt to the non-availability of resources or even appearances of new devices.
\end{description}

\section{Scope}\label{sec:scope}

The focus of this dissertation is the development of a methodology and prototype that allows for a decentralized orchestration of an IoT system. Despite security being a critical feature, it is considered a secondary goal. The scope of the project is a home automation system, where its devices are running the developed firmware, based on MicroPython firmware. No modification will be made to the visual editor of Node-RED.

\section{Main Hypothesis}\label{sec:main_hypothesis}

This dissertation is built around the following hypothesis:

\begin{quote}
    \emph{``Given an IoT system with several heterogeneous devices connected, capable of running custom code, a decentralized architecture is more resilient, efficient and scalable than a centralized one.''}
\end{quote}

The attributes presented in the hypothesis will be measure against a system using the current development branch of Node-RED. These attributes consist of:

\begin{itemize}
    \item \textbf{Resilience} means the system's capability to adapt to failures and changes. It will be measured by injecting failures and measuring the recovery patterns.
    \item \textbf{Efficiency} how fast the system can execute the logic of the system and communicate between nodes. The efficiency will be measured by the latency taken to react to certain events.
    \item \textbf{Elasticity} specifies how a system can grow and shrink. This attribute will be tested by increasingly adding or removing devices in different scenarios and assessing the overall system's behavior.
\end{itemize}

\section{Experimental Methodology}\label{sec:exp_meth}

In the interest of validating whether or not the solution implemented achieves the \emph{desiderata} and solves the current issues, we present test scenarios and controlled experiments that use both virtual and physical devices. Each of these scenarios will measure one or more requirements proposed in Section \ref{sec:desiderata}. The attributes mentioned in Section \ref{sec:main_hypothesis} will then be evaluated against the implemented solution, in order to assess our main hypothesis.

\section{Summary}\label{sec:stat_summary}

\sectionref{sec:current_issues} starts by exposing the issues and lack of features not fully implemented in the current tools presented in \chapterref{chap:sota}. \sectionref{sec:desiderata} defines a \textit{desiderata} that aims to fix the issues presented in \sectionref{sec:current_issues}. The hypothesis of this dissertation is detailed in \sectionref{sec:main_hypothesis}, as well as an experimental methodology to prove it, in \sectionref{sec:exp_meth}.
