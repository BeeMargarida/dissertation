\chapter{Introduction} \label{chap:intro}

\section*{}

This chapter introduces the motivation and scope of this project, as well as the problems it aims to solve. Section \ref{sec:context} details the context of this project in the area it is based on. Section \ref{sec:motivation} explains the reason why this work and the area it belongs to is important. Then, section \ref{sec:problem_definition} defines the problem we aim to solve and the goals of this dissertation are described in section \ref{sec:goals}. Finally, the section \ref{sec:document structure} describes the structure of this document and what content it contains.

\section{Context} \label{sec:context}

The Internet of Things paradigm states that all devices, independently of their capabilities, are connected to the Internet and allow for the transfer, integration and analytic of data generated by them \cite{Buyya:2016:ITP:3050877}. This paradigm has several characteristics, such as the heterogeneity and high distribution of devices as well as their increasing connectivity and computational capabilities \cite{SoS}. All this factors allow for a great level of applicability, enabling the realization of systems for management of cities, health services and industries \cite{6851114}.
\par The interest in Internet of Things has been growing massively, following the rising of connected devices along these past years. According to Siemens, in 2020 there will be around 26 billion physical devices connected to the Internet and in 2025 the predictions are pointing at 75 billion \cite{tanweer}. Although this allows for more opportunities, it is important to note that these devices are very different in their hardware and capabilities, which causes several problems in terms of development the systems, as well as their scalability, maintainability and security. 
\par Visual Programming Languages (VPLs) allow the user to communicate with the system by using and arranging visual elements that can be translated into code \cite{vpl-book}. It provides the user with an intuitive and straightforward interface for coding at the possible cost of loosing functionality. There are several programming languages with different focuses, such as education, video game development, 3D building, system design and even Internet of Things \cite{survey_vpl_iot}. Node-RED\footnote{https://nodered.org/} is one of the most famous open source visual programming tool, originally developed by IBM’s Emerging Technology Services team and now a part of the JS Foundation, which provides an environment for users to develop their own Internet of Things systems.

\textcolor{red}{Falar de resiliência e atributos não funcionaus de qualidade de sistemas; falar sobre arquiteturas centralizadas e decentralizadas; mencionar fog e edge}

\section{Problem Definition} \label{sec:problem_definition}

Most mainstream visual programming tools focused on Internet of Things, Node-RED included, have a centralized approach, where a main component executes most of the computation on data provided by edge devices, e.g. sensors and gateways. There are several consequences to this approach: (a) computation capabilities of the edge devices are being ignored, (b) it introduces a single point of failure, and (c) local data is being transferred across boundaries (private, technological, political...) either without need, or even in violation of legal constraints. The principle of Local-First - i.e, data and logic should reside locally, independent of third-party services faults and errors - is being ignored. 
\par Besides being a single point of failure, centralized systems can be less efficient than decentralized ones and in this context it might be the case, since there are computation capabilities that aren't being taken advantage of.
\par Chapter \ref{chap:problem_statement} expands on the problem definition, explaining it in bigger detail, defining its scope, use cases and research questions. \textcolor{red}{\textbf{**CHECK THIS**}}

\section{Motivation} \label{sec:motivation}

\textcolor{red}{\textbf{Still needs something...more references and stuff}}\\
Internet of Things is a rapid growing concept that is being applied to several areas, such as home automation, industry, health, city management and many others. Given the number of existing systems with different protocols and architectures, it becomes difficult for a user to build a system that is in accordance to standards \cite{standard-iot}. 
\par With the appearance of visual programming languages focused in IoT, more specifically Node-RED, users can build their own systems in an easier and streamlined way, removing the overhead of learning advanced programming concepts and protocols.   

\section{Goals} \label{sec:goals}

The main goal of this dissertation is to leverage the computation capabilities of the devices in the network, increasing efficiency, fault-tolerance, resiliency and scalability in an Internet of Things system.
\par To achieve this goal, a prototype will be developed, extending or rewriting Node-RED, that enables IoT devices to communicate their "computational capabilities" back to the orchestrator. In its turn, the orchestrator is able to partition the computation and send "tasks" to the nodes, which are the devices in the network, leveraging their computation power and independence.
\par As a secondary goal, several other challenges will be tackled, viz: (i) inferring computational capabilities of the devices in the network, (ii) detecting non-availability and using alternative computation resources, and (iii) exploring different alternatives of leveraging current IoT devices, including using firmwares that allow the execution of programs written in Lua, Javascript, Python, etc., amongst others.

\section{Document Structure} \label{sec:document structure}

Chapter \ref{chap:background} introduces the background information and explanation about concepts necessary for the full understanding of this dissertation with the use of a Systematic Literature Review on the state of the art of visual programming applied to the Internet of Things paradigm. Chapter \ref{chap:sota} describes the state of the art regarding the ecosystem of this project's scope. Chapter \ref{chap:problem_statement} presents the problem this dissertation aims to solve, as well as the approach taken to solve it. Chapter \ref{chap:solution} details how the solution was implemented and all the decisions and efforts taken to answer the problem statement mentioned before. Chapter \ref{chap:evaluation} analyzes the evaluation process and explains how the solution was validated. Finally, Chapter \ref{chap:concl} concludes the dissertation with a reflection on the success of the project by presenting the a summary of the contributions made and detailing the difficulties and future work.
