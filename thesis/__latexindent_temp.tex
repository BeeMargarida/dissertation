%bibtex entries 


@INPROCEEDINGS{8473371,
    author={N. K. {Giang} and R. {Lea} and M. {Blackstock} and V. C. M. {Leung}},
    booktitle={2018 IEEE International Conference on Edge Computing (EDGE)},
    title={Fog at the Edge: Experiences Building an Edge Computing Platform},
    year={2018},
    volume={},
    number={},
    pages={9-16},
    keywords={cloud computing;Internet of Things;edge computing platform;technology advancement;network edge;communications;edge devices;edge network;context-dependent characteristics;application logic;large-scale IoT applications;distributed node-RED;open source node-RED tool;Cloud computing;Wires;Programming;Edge computing;Computational modeling;Buildings;Robots;edge;fog computing;exogenous;dataflow;coordination},
    doi={10.1109/EDGE.2018.00009},
    ISSN={null},
    month={July},
}

@ARTICLE{8374025,
    author={N. K. {Giang} and R. {Lea} and V. C. M. {Leung}},
    journal={IEEE Access},
    title={Exogenous Coordination for Building Fog-Based Cyber Physical Social Computing and Networking Systems},
    year={2018},
    volume={6},
    number={},
    pages={31740-31749},
    keywords={cyber-physical systems;data flow computing;distributed programming;embedded systems;mobile computing;social networking (online);social networks;CPSCN systems;cyber physical social computing-and-networking systems;mobile phones;smart vehicles;computation activities;distributed dataflow programming model;application platform;social applications;physical world;fog computing systems;network access points;heterogeneous devices;distributed devices;cloud-based infrastructure;networking systems;smart embedded devices;communication activities;exogenous coordination model;computing resources;Edge computing;Computational modeling;Vehicle dynamics;Social network services;Cloud computing;Buildings;Context modeling;Cyber physical social computing and networking;fog computing;Internet of Things;exogenous coordination;data-flow},
    doi={10.1109/ACCESS.2018.2844336},
    ISSN={2169-3536},
    month={},
}

@article{SoS,
    author={Fahed Alkhabbas and Romina Spalazzese and Paul Davidsson},
    journal={Proceedings of the 2nd edition of Swedish Workshop on the Engineering of Systems of Systems (SWESOS 2016)},
    title={IoT-based Systems of Systems},
    year={2016}
}

@ARTICLE{6851114,
    author={S. {Chen} and H. {Xu} and D. {Liu} and B. {Hu} and H. {Wang}},
    journal={IEEE Internet of Things Journal},
    title={A Vision of IoT: Applications, Challenges, and Opportunities With China Perspective},
    year={2014},
    volume={1},
    number={4},
    pages={349-359},
    keywords={Internet of Things;research and development;China;Internet of Things;IoT development;R&D plans;IoT architecture;Standards;Industries;Vehicles;Computer architecture;Sensors;Monitoring;Internet of Things (IoT);IoT application;IoT architecture;IoT challenge;IoT standardization},
    doi={10.1109/JIOT.2014.2337336},
    ISSN={2372-2541},
    month={Aug},
}

@ARTICLE{ISOIEC,
author={ISO/IEC  JTC  1},
title={Internet  of  things  (iot)  -  preliminary  report},
journal={ISO,Tech. Rep.},
year={2014},
}

@article{tanweer,
author = {Alam, Tanweer},
year = {2018},
month = {05},
pages = {},
title = {A Reliable Communication Framework and Its Use in Internet of Things (IoT)},
volume = {3}
}

@INPROCEEDINGS{standard-iot,
author={S. A. {Al-Qaseemi} and H. A. {Almulhim} and M. F. {Almulhim} and S. R. {Chaudhry}},
booktitle={2016 Future Technologies Conference (FTC)},
title={IoT architecture challenges and issues: Lack of standardization},
year={2016},
volume={},
number={},
pages={731-738},
keywords={Internet of Things;standardisation;IoT architecture;standardization;Internet of Things;connected things heterogeneity;IoT;IoT Architecture;Machine to Machine (M2M);IoT Standardization;Thread and 6LoPAN},
doi={10.1109/FTC.2016.7821686},
ISSN={null},
month={Dec},}


@article{Ray2017,
  doi = {10.1155/2017/1231430},
  url = {https://doi.org/10.1155/2017/1231430},
  year = {2017},
  publisher = {Hindawi Limited},
  volume = {2017},
  pages = {1--6},
  author = {Partha Pratim Ray},
  title = {A Survey on Visual Programming Languages in Internet of Things},
  journal = {Scientific Programming}
}

@book{vpl-book,
author = {Chang, S K},
title = {Handbook of Software Engineering and Knowledge Engineering},
publisher = {World Scientific Publishing Company},
year = {2002},
doi = {10.1142/4603-vol2},
address = {},
edition   = {},
URL = {https://www.worldscientific.com/doi/abs/10.1142/4603-vol2},
eprint = {https://www.worldscientific.com/doi/pdf/10.1142/4603-vol2}
}

@article{SLR_guidelines,
title = "Guidelines for conducting systematic mapping studies in software engineering: An update",
journal = "Information and Software Technology",
volume = "64",
pages = "1 - 18",
year = "2015",
issn = "0950-5849",
doi = "https://doi.org/10.1016/j.infsof.2015.03.007",
url = "http://www.sciencedirect.com/science/article/pii/S0950584915000646",
author = "Kai Petersen and Sairam Vakkalanka and Ludwik Kuzniarz",
keywords = "Systematic mapping studies, Software engineering, Guidelines",
abstract = "Context
Systematic mapping studies are used to structure a research area, while systematic reviews are focused on gathering and synthesizing evidence. The most recent guidelines for systematic mapping are from 2008. Since that time, many suggestions have been made of how to improve systematic literature reviews (SLRs). There is a need to evaluate how researchers conduct the process of systematic mapping and identify how the guidelines should be updated based on the lessons learned from the existing systematic maps and SLR guidelines.
Objective
To identify how the systematic mapping process is conducted (including search, study selection, analysis and presentation of data, etc.); to identify improvement potentials in conducting the systematic mapping process and updating the guidelines accordingly.
Method
We conducted a systematic mapping study of systematic maps, considering some practices of systematic review guidelines as well (in particular in relation to defining the search and to conduct a quality assessment).
Results
In a large number of studies multiple guidelines are used and combined, which leads to different ways in conducting mapping studies. The reason for combining guidelines was that they differed in the recommendations given.
Conclusion
The most frequently followed guidelines are not sufficient alone. Hence, there was a need to provide an update of how to conduct systematic mapping studies. New guidelines have been proposed consolidating existing findings."
}


@inbook{inbook,
    author = {Sendorek, Joanna and Szydlo, Tomasz and Windak, Mateusz and Brzoza-Woch, Robert},
    year = {2019},
    month = {06},
    pages = {634-647},
    title = {FogFlow - Computation Organization for Heterogeneous Fog Computing Environments},
    doi = {10.1007/978-3-030-22744-9_49}
}

@misc{Node-red2017,
    author = {Node-red},
    isbn = {9781509302062},
    pages = {1--5},
    title = {{Flow-based programming for the Internet of Things}},
    url = {https://nodered.org/},
    year = {2017},
    howpublished = "\url{https://nodered.org/}",
    note = "[Online; accessed November 2019]"
}
% check if this one is needed - definition of IoT
@book{Buyya:2016:ITP:3050877,
 author = {Buyya, Rajkumar and Dastjerdi, Amir Vahid},
 title = {Internet of Things: Principles and Paradigms},
 year = {2016},
 isbn = {012805395X, 9780128053959},
 edition = {1st},
 publisher = {Morgan Kaufmann Publishers Inc.},
 address = {San Francisco, CA, USA},
} 