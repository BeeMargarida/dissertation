\chapter{Background} \label{chap:background}

\section*{}


This chapter describes the necessary foundations regarding visual programming tools for the Internet of Things context. Secion \ref{sec:background_iot} describes the background of the Internet of Things paradigm and important concepts in that area. Section \ref{sec:background_vpl} mentions visual programming languages, their uses as well as their benefits and drawbacks.  \textcolor{yellow}{\textbf{**TODO**}}

\section{Internet of Things}\label{sec:background_iot}

Internet of Things paradigm was defined by the committee of the International Organization for Standardization and the International Electrotechnical Commission \cite{ISOIEC} as:
\begin{quote}
    “An infrastructure of interconnected objects, people, systems and information resources together with intelligent services to allow them to process information of the physical and the virtual world and react.”
\end{quote}
\par This paradigm is built upon the network of heterogenous devices interconnected between themselves, people and the environment. According to Buuya \cite{iot_future_direction}, the applications of IoT systems can be divided into four categories: (i) Home at the scale of an individual or home, (ii) Enterprise at the scale of a community, (iii) Utilities at a national or regional scale and (iv) Mobile, which is spread across domains due to its large scale in connectivity ans scale. 
%Nowadays, the use of Internet of Things systems is present in different areas, such as aerospace, automotive, telecommunications and health industries, as well as city and agriculture managements, amongst others \cite{applications_iot}.
\par However, one might think that IoT only relates to machines and interactions between them. Most of the devices we use in our day-to-day - mobile phones, security cameras, watches, coffee machines - are now computation capable of making moderately complex tasks and are constantly generating and sending information, some of it to their users. This relates to the \emph{human-in-the-loop} concept, where humans and machines have a symbiotic relationship \cite{human_in_the_loop_survey}.
 
\subsection{IoT architectures}\label{sec:architectures}

\textcolor{blue}{Talk about the three types of architectures - cloud computing, fog computing, edge computing; maybe insert a nice image}

\textcolor{blue}{Start by talking about the starting architecture: cloud computing, what it is, drawbacks - latency and huge use of bandwidth; Explain 2 alternatives: Fog Computing and Edge Computing. Check sources from link below}

%https://en.wikipedia.org/wiki/Fog_computing

\subsection{Applications}\label{sec:iot_applications}

\textcolor{red}{Not sure if this one is needed}

\section{Visual Programming Languages}\label{sec:background_vpl}

\textcolor{blue}{explain what a vpl is; what is its goal; why it is good, its drawbacks; characteristics of vpls; classification of VPLs}

\section{Related Work}

\textcolor{red}{Is this one needed? who knows}

\section{Summary}

\textcolor{yellow}{\textbf{**TODO**}}
