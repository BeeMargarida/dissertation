\chapter{Evaluation} \label{chap:evaluation}

\section*{}

\textbf{Environmental Setups}:
\begin{enumerate}
    \item A room has 3 sensors that give temperature and humidity readings every minute. There’s a virtual sensor that compares the results (of both temperature and humidity) and triggers depending on some configured thresholds. An AC uses those readings to decide (a) if it switches on/off, (b) its operating mode: cool, heat, and dehumidify. The Minimal Working System (MWS) consists in (a) one temperature sensor, (b) one humidity sensor, (c) one node capable of making the decision, and (d) working communication channels amongst them
    \item In a scenario with an increasing number of devices, possible logarithmic, where each node receives an input, executes simple calculations and sends an output. It can have multiple starting points and several topologies.
\end{enumerate}

\textbf{Experimental Setup(s)}:
\begin{enumerate}
    \item Assess that the Minimal Working System (MWS) is achieved:
        \begin{enumerate}
            \item \textbf{Sanity check.} All tasks are simple readings and forwarding, no compensation or other fault-tolerance strategy. Each sensor does its own thing. Orchestration is centralized. We expect all roundtrips to take less than the smallest part that can be resolved (measurement capability, which we estimate to be <1s).
            \item \textbf{Re-orchestration.}
                \begin{enumerate}
                    \item MWS is achieved via multiple possible configurations by selective (provoked) device failure (fail-stop);
                    \item Inconsistent device behaviour, e.g. appear and disappear in shorter intervals lower that the time needed for orchestrating convergence (OCT), that leads to activity impacting the MWS;
                \end{enumerate}
        \end{enumerate}
    \item Using all participant devices and nodes in the experimental setup:
        \begin{enumerate}
            \item \textbf{Sanity check.} All tasks are simple readings and forwarding, no compensation or other fault-tolerance strategy.
            \item \textbf{Re-orchestration.}
                \begin{enumerate}
                    \item \textbf{Restrictions (predicates) are enforced.} Check that possible configurations lead to solutions that enforce defined predicates;
                        \begin{enumerate}
                            \item Temperature and humidity might coexist in the same, or in dedicated, devices;
                        \end{enumerate}
                    \item \textbf{Priorities are honored.} Check that all specified priorities were taken into account, and only violated if necessary;
                        \begin{enumerate}
                            \item Priority is given to edge devices, but fog and cloud can be used;
                            \item Priority is given to the maximum level of decentralization, but some centralization can be used.
                        \end{enumerate}
                    \item \textbf{Out of memory fail-safe}
                    \item \textbf{Memory leaks}
                    \item \textbf{Latency.} Make devices selectively slow and check the consequences; might impact OCT and MWS.
                \end{enumerate}
        \end{enumerate}
\end{enumerate}




% node
% bestIndex = 0

% for device in devices:
%     if not all node predicates in device tags: return
%     intersectionIndex = (nº of node priorities in device tags)/(nº node priorities)
    
%     matchIndex = 
%         intersectionIndex * 0.5 + 
%         (1/( nodes assigned to the device) + 1) * 0.4 +
%         (nº of node priorities in device tags/ device tags) * 0.1
    
%     if matchIndex > bestIndex:
%         bestIndex = matchIndex
%         device is the best choice for node

% \section{Assessment}

% \section{Research Questions}

% \section{Conclusions}