\chapter{Evaluation} \label{chap:evaluation}

\section*{}

\textbf{Environmental Setups}:
\begin{enumerate}
    \item A room has 3 sensors that give temperature and humidity readings every minute. There’s a virtual sensor that compares the results (of both temperature and humidity) and triggers depending on some configured thresholds. An AC uses those readings to decide (a) if it switches on/off, (b) its operating mode: cool, heat, and dehumidify. The Minimal Working System (MWS) consists in (a) one temperature sensor, (b) one humidity sensor, (c) one node capable of making the decision, and (d) working communication channels amongst them
    \item In a scenario with an increasing number of devices, possible logarithmic, where each node receives an input, executes simple calculations and sends an output. It can have multiple starting points and several topologies.
\end{enumerate}

\textbf{Experimental Setup(s)}:
\begin{enumerate}
    \item Assess that the Minimal Working System (MWS) is achieved:
        \begin{enumerate}
            \item \textbf{Sanity check.} All tasks are simple readings and forwarding, no compensation or other fault-tolerance strategy. Each sensor does its own thing. Orchestration is centralized. We expect all roundtrips to take less than the smallest part that can be resolved (measurement capability, which we estimate to be <1s).
            \item \textbf{Re-orchestration.}
                \begin{enumerate}
                    \item MWS is achieved via multiple possible configurations by selective (provoked) device failure (fail-stop);
                    \item Inconsistent device behaviour, e.g. appear and disappear in shorter intervals lower that the time needed for orchestrating convergence (OCT), that leads to activity impacting the MWS;
                \end{enumerate}
        \end{enumerate}
    \item Using all participant devices and nodes in the experimental setup:
        \begin{enumerate}
            \item \textbf{Sanity check.} All tasks are simple readings and forwarding, no compensation or other fault-tolerance strategy.
            \item \textbf{Re-orchestration.}
                \begin{enumerate}
                    \item \textbf{Restrictions (predicates) are enforced.} Check that possible configurations lead to solutions that enforce defined predicates;
                        \begin{enumerate}
                            \item Temperature and humidity might coexist in the same, or in dedicated, devices;
                        \end{enumerate}
                    \item \textbf{Priorities are honored.} Check that all specified priorities were taken into account, and only violated if necessary;
                        \begin{enumerate}
                            \item Priority is given to edge devices, but fog and cloud can be used;
                            \item Priority is given to the maximum level of decentralization, but some centralization can be used.
                        \end{enumerate}
                    \item \textbf{Out of memory fail-safe}
                        \begin{itemize}
                            \item \textbf{Scenario A.} With 20 devices, each one with different processing capabilities. During orchestration, some devices will develop an out-of-memory error because they can't process all the processing tasks assigned to them, specifically the size of the script given. The orchestrator decides to send less tasks to these devices. The system will converge in a working solution. \textit{This scenario will be implemented with a modified device script. When devices receive a script, it will generate errors if the size of the script passes a specific threshold, which will be different between devices. This simulates the memory constraints of devices when receiving tasks.}
                            \item \textbf{Scenario B.} With 20 devices, some of them have a memory leak with an unknown cause. After random time Random(t0,t1), these problematic devices stop working with an out-of-memory error. The orchestrator thinks that the devices can't handle the quantity of processing tasks assigned to them, so in the re-orchestration it will assign fewer tasks. Since these devices will always break, the orchestrator will eventually not consider these devices in the assignment of nodes. \textit{This scenario will be implemented with a modified device script that will trigger an out-of-memory error after a random period after executing the given tasks.}
                            \item \textbf{Scenario C.} With 20 devices, there is a device that is sensitive to a particular node, which causes the device to give out an out-of-memory error. The orchestrator will potentially assign this node to the specific device. When the device gives out the out-of-memory error, the orchestrator will eventually converge in a solution where the node is not assigned to the particular device, and the system will converge.  \textit{These out-of-memory errors will be simulated with the use of a failure node that forces an \texttt{MemoryException} in the device.}
                        \end{itemize}
                    \item \textbf{Latency.} Make devices selectively slow and check the consequences; might impact OCT and MWS. ?
                \end{enumerate}
        \end{enumerate}
\end{enumerate}




% node
% bestIndex = 0

% for device in devices:
%     if not all node predicates in device tags: return
%     intersectionIndex = (nº of node priorities in device tags)/(nº node priorities)
    
%     matchIndex = 
%         intersectionIndex * 0.5 + 
%         (1/( nodes assigned to the device) + 1) * 0.4 +
%         (nº of node priorities in device tags/ device tags) * 0.1
    
%     if matchIndex > bestIndex:
%         bestIndex = matchIndex
%         device is the best choice for node

% \section{Assessment}

% \section{Research Questions}

% \section{Conclusions}