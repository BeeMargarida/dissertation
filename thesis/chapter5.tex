\chapter{Solution} \label{chap:solution}

\section*{}

\subsection{Challenges}
\begin{itemize}
    \item Problemas de memória e espaço dos ESPs para execução de scripts de micropython com 3+ nós (ESPs lançam erros de alocamento de memória se o script enviado é maior que um certo nº de bytes ou ao fazer redeploy de scripts de médio tamanho). - Alternativas: failsafe (implementado e funcional), pyc (não existe para micropython mas existe o .mpy. No entanto, precisa de ser executado para gerar um .mpy a partir de um .pt, o que não se aplica à solução atual), ota (não existem boas soluções para micropython)
    \item Necessidade de implementar novos nós para situações de teste
    \item Node-red não suporta comunicação de mqtt entre nós de raíz. Node-red teve de ser adaptado para que a comunicação entre nós fosse feita desta maneira.
    \item Modificar scritps e suporte para o port para Unix do micropython - muitas diferenças, limitações e criação de muitos bugs.
    \item Limitações das bilbiotecas usadas de micropython, especificamente nas bibliotecas de mqtt e operações assíncronas.
\end{itemize}{}
