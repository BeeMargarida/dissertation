\documentclass[AMA,STIX1COL]{WileyNJD-v2}

\usepackage{booktabs}
\usepackage{listings}
\usepackage{threeparttable}
\graphicspath{{figures/}}

\renewcommand{\thefootnote}{\fnsymbol{footnote}}

\articletype{Research Article}%

\received{}
\revised{}
\accepted{}
        
\raggedbottom

\begin{document}

\title{Decentralized Orchestration in IoT with Visual Programming: A Systematic Literature Review}

\author[1]{Margarida Silva}
\author[1,2]{João Pedro Dias*}
\author[1,3]{André Restivo}
\author[1,2]{Hugo Sereno Ferreira}

\authormark{Margarida Silva \textsc{et al}}


\address[1]{\orgdiv{DEI}, \orgname{Faculty of Engineering, University of Porto}, \orgaddress{\state{Porto}, \country{Portugal}}}
\address[2]{\orgname{INESC TEC}, \orgaddress{\state{Porto}, \country{Portugal}}}
\address[3]{\orgname{LIACC}, \orgaddress{\state{Porto}, \country{Portugal}}}

\corres{*João Pedro Dias, DEI - FEUP, Rua Dr. Roberto Frias, Porto, Portugal \email{jpmdias@fe.up.pt}}

%\presentaddress{This is sample for present address text this is sample for present address text}

\abstract[Summary]{Internet-of-Things (IoT) systems are considered one of the most notable examples of complex, large-scale systems. Some authors have proposed visual programming solutions to address part of this inherent complexity, but most of these systems depend on a centralized unit to carry out most -- if not all -- the orchestration between devices and system components, hindering the degree of scalability and distribution that can be attained. In this work, we carry out a systematic literature review of the current solutions that provide visual and decentralized orchestration to define and operate IoT systems. Our work reflects upon a total of 29 proposals that address these issues up to a certain degree. We provide an in-depth discussion of these works, and find out that only four of these solutions attempt to tackle this issue as a whole, though still leaving a set of open research challenges. We finally argue that this challenges, if addressed, could make IoT systems more fault-tolerant, with impact on their dependability, performance, and scalability.}

\keywords{Internet-of-Things, Orchestration, Visual Programming, Decentralized Computation, Large-Scale Systems}

\maketitle

\input{parts/1-introduction}

\input{parts/2-background}

\input{parts/3-methodology}

\input{parts/4-results}

\input{parts/5-discussion}

\input{parts/6-conclusions}


\bibliography{wileyNJD-AMA}

\end{document}
