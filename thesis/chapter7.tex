\chapter{Conclusions} \label{chap:concl}

\section*{}

As the number of devices connected to the internet increases, it is important to leverage their capabilities and modify the way systems are build in order to take advantage of this resources. It is also important to allow end users with no programming experience to build IoT systems, with the use of visual programming tools. 

During the analysis of the state of the art, some issues and missing features were identified, which this dissertation aims to correct. Despite the existence of a considering number of visual programming tools applied to IoT, the majority of these tools are centralized. The ones that have a decentralized architecture have limiting characteristics such as assumptions about what is a constrained device regarding computational capabilities, lack of open source licenses and simplification of the approach taken to the decomposition and assignment of tasks.

This dissertation aims to solve these issues by expanding an already popular visual programming tool, Node-RED, with a decentralized approach that focuses of leveraging all the devices, even ones with support only for execution of simple blocks of code. The expected result is a decentralized system, that can self-adapt to runtime conditions, and decomposes the given computations into independent tasks, which are assigned to devices. This assignment is made in order to increase the efficiency of the system, reducing latency and distributing CPU usage.  


%\section{Difficulties}

%\section{Contributions}

%\section{Conclusions}

%\section{Future Work}