\chapter{Conclusions} \label{chap:concl}

\section*{}

\minitoc \mtcskip \noindent
This chapter presents an overview of this dissertation. \sectionintroref{sec:difficulties} describes the difficulties faced during the development of the solution. \sectionintroref{sec:future_work} lists avenues that were not explored during the development phase and improvements to the final solution. Lastly, \sectionintroref{sec:conclusion_conclusions} contains an overview of the developed work.

\section{Difficulties}\label{sec:difficulties}

During the development of the solution, there were several challenges to overcome to build the solution that better fitted the proposed requirements. Some of these difficulties were mentioned in previous sections, some were solved and others will be mentioned in \sectionref{sec:future_work}.

The first difficulty was the modification of communication between Node-RED \textit{nodes}. Since they originally used events, a solution exclusive to Javascript, and a centralized environment, changes were made to implement a decentralized alternative. One of the most challenging parts of this feature was the implementation of this communication for \emph{sub-flows}. The way Node-RED implements this entity made it difficult to translate to the new way of communication. This was implemented but was seldom used in the constructed scenarios.

There were several difficulties when it comes to the devices' firmware. Early on in the development phase, it was noted the memory limitations of ESP8266 chips, which led to the construction of the \textsc{fail-safe} mechanism, after exploring several others. However, it was noted that if an assigned script passed a certain threshold, the device could not recover. This led to the exploration of several alternatives until the change in hardware from ESP8266 to ESP32. Besides this difficulty, there were some problems regarding the differences between MicroPython Unix and ESP ports. This led to several changes in the developed firmware to allow compatibility in both environments. 

Finally, the libraries used in the MicroPython firmware have some limitations, which were mentioned before in \sectionref{sec:devices_decentralization}. These limitations consist of the MQTT client's inexistence of support for QoS.

\section{Future Work}\label{sec:future_work}

The solution developed during the course of this dissertation solved the more pressing issues identified in \sectionref{sec:current_issues}. However, the implementation contains limitations and introduces new problems, which can be expanded upon and solved in future work.

As mentioned previously in \sectionref{sec:evaluation_discussion} experiments, the orchestrator is sensible to any change in the status of the devices. This characteristic leads the orchestrator to perform several orchestrations that are costly to the system. For example, one failed \textsc{Ping} request should not lead to a complete (re-)orchestration of the system, since the device non-response may not mean total unavailability. Instead, there should be retries, where the system made sure that the device was indeed unavailable. This mechanism should be configurable by the user or based on existing algorithms used in distributed systems.

Currently, every time the orchestrator starts the assignment process, the system stops working until the process is finished. One optimization that would greatly increase the system's availability would be to implement a way of (re)orchestrating without forcing the system to stop. For example, instead of sending a script that contains all the \textit{nodes}' code, send snippets for each node and each device would be responsible for adding it to its execution script. However, this is not a trivial problem and more exploration would have to be made.

The \textit{node} assignment method used in the developed solution uses a greedy algorithm to assess the best device for each \textit{node}. As mentioned in \sectionref{sec:node_red_computation_decentralization}, this has limitations that can lead to impossible assignments. The assignment process could be greatly improved with the use of better algorithms, for example by using SAT solving algorithms. This improvement can lead to several different solutions, each with their advantages.

Lastly, to validate the current solution, only a small portion of \textit{nodes} were developed that support MicroPython code generation. Besides the increase in the number of nodes that support this, it would be interesting if the system was expanded to support different types of devices' firmware. For example, supporting code generation for C and Lua.

We can conclude that the developed tool has space for improvement, not only in the expansion to new firmware and environments but also in its optimization and enrichment.

% \section{Contributions}\label{sec:contributions}

% To solve the issues detailed in Section \ref{sec:current_issues} and to implement the features proposed in Section \ref{sec:desiderata}, this work presents several contributions, which are as follow:

% \begin{description}
%     \item [Systematic Literature Review of Visual Programming and Orchestration in the Internet-of-Things]: 
%     \item [Decentralized Node-RED]
%     \item [MicroPython firmware for decentralization support] so that each computational task can be executed in edge and fog devices, which contain limited resources.
%     \item [D4: Provide self-adaptation of the system] so that it can adapt to the non-availability of resources or even appearances of new devices.
% \end{description}

% \textcolor{blue}{Complete the conclusion with the conclusions from the evaluation}

\section{Conclusions}\label{sec:conclusion_conclusions}

As the number of devices connected to the internet increases, it is important to leverage their capabilities and modify the way systems are built to take advantage of these resources. It is also important to allow end-users with no programming experience to build Internet-of-Things (IoT) systems, with the use of visual programming tools. These tools make the building process easier, reducing the need for knowledge of programming concepts.

Despite the existence of a considering number of visual programming tools applied to IoT, the majority of these tools are centralized. This centralization hinders the resiliency of the system, as the unit responsible for the execution of most or all of the computation is a single point of failure. If this unit or the network fails, the system stops being functional. In addition to this, there are several computational resources in the devices that are not being utilized by the system.

During the analysis of the state of the art, some issues and missing features were identified, which this dissertation aims to correct. The tools found that possess a decentralized architecture have limiting characteristics such as assumptions about what is a constrained device regarding computational capabilities, lack of open source licenses, and simplification of the approach taken to the decomposition and assignment of tasks.

The developed solution solves these issues by expanding an already popular visual programming tool, Node-RED, with a decentralized approach that focuses on leveraging all the devices, even ones that only support the execution of simple blocks of code. Node-RED was modified to allow communication between \textit{nodes}, even in different devices, as well as support for \textit{node}'s code generation and orchestration of computations. On the devices' side, firmware was developed that allows it to receive scripts of code for later execution as well as communicate its status. 

Several mechanisms were built to deal with the devices' instability. The developed solution manages the state of the devices, triggering a new orchestration if any device becomes unavailable or if a new device announces itself. Besides this, the system handles possible memory constraints of the system, assigning fewer \textit{nodes} to devices limited by memory capacity.

This dissertation contributes with a decentralized IoT system that is robust, elastic, and overall efficient.